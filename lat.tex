\documentclass{article}
\usepackage{graphicx} % Required for inserting images
\usepackage{hyperref}
\usepackage{minted}
\usepackage[letterpaper, top=2cm, bottom=3cm, left=3cm, right=3cm, heightrounded]{geometry}

\title{IT - Security, Mandatory II}
\author{Rasmus Carlsen, Lucas Modin & Lukas Balthazar}
\date{November 2025}

\begin{document}

\maketitle

\section*{Problem 1: SQL Injection (Software Security)}
SQLI are a way to insert malicious query code in a input field in apps, so one can manipulate the app to show the data in the database or altering the database, eg. deleting the whole database or rows/columns, adversaries would maybe have some reason for altering.
In the \href{www.portswigger.net}{Portswigger.net} under the learning path for academy and in the sql path, we can find information on the sql injection attacks.
Here labs are provided for exercising sql attacks.

\renewcommand{\thesubsection}{\Alph{subsection}}
\subsection{Using the 'Union select NULL...' command}

To use the 
\begin{minted}{sql}
    UNION SELECT NULL, NULL, NULL;
\end{minted} 
sql attack, we need to use the websites url for injection. Here we would go around the site to see what queries the website does normally, here we can see "web-security-academy.net/filter?category=Gifts" if we click on the Gifts category, now we know that if we can input ' in after the ?category=' and get internal server error that it is vulnerably to sql injections. We now only need to do: UNION SELECT NULL, NULL, NULL... until we get the number of rows/columns that is in the table. So the correct sql injection would be: \verb|?category=Gifts'+UNION+SELECT+NULL,+NULL,+NULL--|, now we know the number of columns/rows in the table, and we can now try to insert to see what datatype each row/column takes, here we try for strings and try with each NULL to set to eg. 'abc', for seing which takes a string:\\ \verb|?category=Gifts'+UNION+SELECT+'abc',+NULL,+NULL--|\\
\verb|?category=Gifts'+UNION+SELECT+'NULL',+abc,+NULL--|\\
\verb|?category=Gifts'+UNION+SELECT+'NULL',+NULL,+abc--|\\
Pros with union is: that we know that that we can find the number of columns, we can find the specific datatype for each column, and we know that if it works that union is permitted.

\begin{figure}[htbp]
  \centering
  \begin{minipage}{0.48\textwidth}
    \centering
    \includegraphics[width=\textwidth,height=4.5cm]{pics/port1.png}
    \caption{Websites main products site}
  \end{minipage}
  \hfill
  \begin{minipage}{0.48\textwidth}
    \centering
    \includegraphics[width=\textwidth,height=4.5cm]{pics/port2.png}
    \caption{After using UNION}
  \end{minipage}
\end{figure}

\subsection{Using the 'Order by' command}
To use the (n == number)
\begin{minted}{sql}
    ORDER BY n;
\end{minted} 
sql attack, we nned to use the websites url for injection as before. This time we would instead in the url type:\\
\verb|category=Gifts'+ORDER+BY+1--|\\
\verb|category=Gifts'+ORDER+BY+2--|\\
\verb|category=Gifts'+ORDER+BY+3--|\\

\begin{figure}[htbp]
  \centering
  \begin{minipage}{0.48\textwidth}
    \centering
    \includegraphics[width=\textwidth,height=4.5cm]{pics/port3.png}
    \caption{Order by 2, for the string column(ascending alphabetically)}
  \end{minipage}
  \hfill
  \begin{minipage}{0.48\textwidth}
    \centering
    \includegraphics[width=\textwidth,height=4.5cm]{pics/port4.png}
    \caption{Order by 3, for the price column(ascending numerically)}
  \end{minipage}
\end{figure}

\noindent Pros with order by, is that it is very fast to get a idea of how many columns the table has, instead of writing a bunch of NULL's in the union. Plus order by may not be restricted in the WAF\cite{cloudflareWAF}

\subsection{Using 'ORDER BY' \& 'UNION' To retrive data from other tables}
We will now use both methods to retrive data from another table, which is hidden in the database. First we will use the 'ORDER BY 2 ...' query to see how many colums that the table has, and after we would use the 'UNION select NULL, NULL' to see how many of the columns are string/varchar values.
\begin{figure}[htbp]
  \centering
  \begin{minipage}{0.48\textwidth}
    \centering
    \includegraphics[width=\textwidth,height=4.5cm]{pics/port5.png}
    \caption{Order by 2;}
  \end{minipage}
  \hfill
  \begin{minipage}{0.48\textwidth}
    \centering
    \includegraphics[width=\textwidth,height=4.5cm]{pics/port6.png}
    \caption{Union SELECT 'abc', 'abc';}
  \end{minipage}
\end{figure}
\newpage
\noindent We now know that the site is vulnerable to sql injections, with unions and we also now know the number of columns and which columns take strings as inputs.\\
The lab gave us hints of what the other tables names was: \verb|users| \& that the columns names was: \verb|username| \& \verb|password|.\\
Now we will use this combined info with the knowledge that there is 2 colums, and we know we can use 'UNION' on the website.
\begin{minted}{sql}
    UNION SELECT username, password FROM users--;
\end{minted}
Now take a close look at the SQLI above and remember the union attack from earlier. You will see they look much alike, username and password are now what is inside the NULL, NULL, and we know another table called users, thats why we can use FROM users;
\begin{minted}{sql}
    UNION SELECT NULL, NULL--;  
\end{minted}
This sqli will show us the table users, where username and password are shown:
\begin{figure}[htbp]
  \centering
  \begin{minipage}{0.48\textwidth}
    \centering
    \includegraphics[width=\textwidth,height=4.5cm]{pics/port7.png}
    \caption{Using the UNION to get another tables dataimage}
  \end{minipage}
  \hfill
  \begin{minipage}{0.48\textwidth}
    \centering
    \includegraphics[width=\textwidth,height=4.5cm]{pics/port8.png}
    \caption{Logged successfully into the administrator account}
  \end{minipage}
\end{figure}
\section{Problem 2. Hashing \& Password Storage (System Security)}
Eve wants to alter Alices message to her boyfriend. 
Since AloveB hash function is describe as vulnerable, she can than use a collision attack. 
We can asume since the hash function is weak, it might be MD5 or Sha1.

\subsection{What is the specific name of this attack? How can this data modification affect
Alice and Bob's relationship?}
The attack is known as a collision attack, by finding two different message that produces the same hash.
If eve can find a different message with an identical hash, she can replace the original message, without Bob detecting any changes.
This can be done by using a birthday attack, which states that two different inputs will eventually give the same output, breaking the collision resistence property.
This could impact Alice and Bobs relationship negatively, and resulting in integrity is broken.


\subsection{What type of suggestions would you give to Alice and Bob to prevent this attack
while using their hash function? (Any advice for more security using the same
hash function?)}
If using the same hash function, we recommend using a bigger output size hashing function eg. minimal sha-256 (32bytes). 
We still highly recommend using a newer hash function like sha3 that is new standard.  
We would also recommend Alice \& Bob to use salting, to add random unpredictable data to the message before hashing. This will reduce collision feasibility.
Secondly Alice \& Bob could, if possible asymmatric cryptography by digitally signing the message with the private key.

\subsection{Considering that Bob works as a system admin in a company, which type of
functions (hash or KDF) would you suggest him to use while storing the
employees' passwords in the database?}
Bob should not use regular hash functions like MD5, SHA-1 or SHA-256 for password storage.
We would recommend using KDF functions like Argon2, Bcrypt and Scrypt. These are designed specifically for storing passwords securely.
But they are slow, hardware heavy and thats makes them take longere to crack, and they are more resistant to brute force attack.


\bibliographystyle{plain}
\bibliography{quotes}
\end{document}